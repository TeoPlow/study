\documentclass[12pt]{article}

\usepackage{fullpage}
\usepackage{multicol,multirow}
\usepackage{tabularx}
\usepackage{listings}
\usepackage{pgfplots}
\usepackage[utf8]{inputenc}
\usepackage[russian]{babel}
\usepackage{pgfplots}
\usepackage{tikz}

% Оригиналный шаблон: http://k806.ru/dalabs/da-report-template-2012.tex

\begin{document}

\section*{Лабораторная работа №9\, по курсу дискрeтного анализа: Графы}

Выполнил студент группы М8О-312Б-22 МАИ \textit{Мамонтов Егор}.

\subsection*{Условие}

\textbf{Вариант:} 6

Поиск кратчайших путей между всеми парами вершин алгоритмом Джонсона

Задан взвешенный ориентированный граф, состоящий из n вершин и m ребер. Вершины пронумерованы целыми числами от 1 до n. Необходимо найти длины кратчайших путей между всеми парами вершин при помощи алгоритма Джонсона. Длина пути равна сумме весов ребер на этом пути. Обратите внимание, что в данном варианте веса ребер могут быть отрицательными, поскольку алгоритм умеет с ними работать. Граф не содержит петель и кратных ребер.
\newpage
\subsection*{Метод решения}

% \newpage
\subsection*{Описание программы}

Для реализации алгоритма были реализованы следующие функции и структуры:
\begin{itemize}
    \item \texttt{void baseSollutions} - функция, которая проходится по базавым случаям. Если подстрока из 1 символа, то она палиндром. Если два соседних символа одинаковые, то палиндромов 3, иначе 2. Для подстрок длиной 3 уже много писанины, поэтому далее идем за помощью к динамическому програмированию.
    \item \texttt{void dynamicProg} - функция, которая сранивает подстроки, длиной более двух символов. Он учитывает кол-во палиндромов во внутренних подстроках и находит кол-во палиндромов в подстроке.
\end{itemize}

\newpage
\subsection*{Дневник отладки}

\begin{enumerate}
    \item Был получен WA на тесте №17. Для решения проблемы нужно было в dynamicProg функции идти от длины подстрок более 2.
    \item Был получен WA на тесте №1. Опечатался.
    \item Был получен OK.
\end{enumerate}

\newpage
\subsection*{Тест производительности}

Сложность алгоритма зависит от сложности заполнения массива, то есть сложность $O(n^2)$, где n - длина входной строки.

\begin{tikzpicture}
\begin{axis}[xlabel={Время, s}, ylabel={Длина строки, символы}]
\addplot coordinates {
    (0.11, 100)
    (0.34, 200)
    (0.48, 300)
    (0.80, 400)
    (1.11, 500)
    (1.42, 600)
    (2.28, 700)
    (3.63, 800)
    (4.41, 900)
    (5.48, 1000)
};
\end{axis}
\end{tikzpicture}


\newpage
\subsection*{Выводы}

Я научился лучше пользоваться динамическим програмированием, учитывая прошлые результаты. Задачу решил с использованием двумерного массива для хранения промежуточных результатов. Это позволило эффективно вычислить количество палиндромов за $O(n^2)$.
\end{document}
