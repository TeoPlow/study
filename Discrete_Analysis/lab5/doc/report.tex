\documentclass[12pt]{article}

\usepackage{fullpage}
\usepackage{multicol,multirow}
\usepackage{tabularx}
\usepackage{listings}
\usepackage{pgfplots}
\usepackage[utf8]{inputenc}
\usepackage[russian]{babel}
\usepackage{pgfplots}
\usepackage{tikz}

% Оригиналный шаблон: http://k806.ru/dalabs/da-report-template-2012.tex

\begin{document}

\section*{Лабораторная работа №5\, по курсу дискрeтного анализа: Суффиксные деревья}

Выполнил студент группы М8О-312Б-22 МАИ \textit{Мамонтов Егор}.

\subsection*{Условие}

\textbf{Вариант:} 2

Найти в заранее известном тексте поступающие на вход образцы с использованием суффиксного массива.

\newpage
\subsection*{Метод решения}

-
% \newpage
\subsection*{Описание программы}

Для реализации алгоритма были реализованы следующие функции:
\begin{itemize}
    \item \texttt{func} - 

\end{itemize}

\newpage
\subsection*{Дневник отладки}

\begin{enumerate}
    \item Был получен TL
\end{enumerate}

\newpage
\subsection*{Тест производительности}

Алгоритм поиска образца в тексте работает за "линейное" время, то есть сложность $O(n+m)$, где n - длина текста.

\begin{tikzpicture}
\begin{axis}[xlabel={Время, ns}, ylabel={Длина текста, строки}]
\addplot coordinates {
    (14799, 1000)
    (198101, 10000)
    (1871960, 100000)
    (18596700, 1000000)
    (181521000, 10000000)
};
\end{axis}
\end{tikzpicture}


\newpage
\subsection*{Выводы}


Я создал ...
\end{document}
