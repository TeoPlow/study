\documentclass[12pt]{article}

\usepackage{fullpage}
\usepackage{multicol,multirow}
\usepackage{tabularx}
\usepackage{listings}
\usepackage{pgfplots}
\usepackage[utf8]{inputenc}
\usepackage[russian]{babel}
\usepackage{pgfplots}
\usepackage{tikz}

% Оригиналный шаблон: http://k806.ru/dalabs/da-report-template-2012.tex

\begin{document}

\section*{Лабораторная работа №4\, по курсу дискрeтного анализа: Поиск образца в строке}

Выполнил студент группы М8О-212Б-22 МАИ \textit{Мамонтов Егор}.

\subsection*{Условие}

\textbf{Вариант:} 0

Необходимо реализовать поиск одного образца в тексте с использованием алгоритма Z-блоков. Алфавит — строчные латинские буквы.

\newpage
\subsection*{Метод решения}

Для написания алгоритма поиска подстроки в строке, сначал нужно было создать Z-функцию. Z-функция для строки s на позиции i равна длине наибольшего подстроки, начинающейся с позиции i, которая является также префиксом строки s.
Далее нужно написать функцию поиска образца. Она использует как раз ранее написанную Z-функцию для поиска. Сначала мы конкатенируем текст и образец, потом применяем к полученной строке Z-функцию и узнаем позиции, где образец совпадает с подстрокой текста.
% \newpage
\subsection*{Описание программы}

Для реализации алгоритма были реализованы следующие функции:
\begin{itemize}
    \item \texttt{z\_function} - Z функция, используется для составления массива хороших суффиксов;
    \item \texttt{find} - поиск подстроки в строке с помощью Z функции;
\end{itemize}

\newpage
\subsection*{Дневник отладки}

\begin{enumerate}
    \item Был получен TL, программа работала медленне, чем надо на 0.1 секунды. Добавил fast\_io для решения проблемы скорости.
\end{enumerate}

\newpage
\subsection*{Тест производительности}

Алгоритм поиска образца в тексте работает за "линейное" время, то есть сложность $O(n+m)$, где n - длина текста.

\begin{tikzpicture}
\begin{axis}[xlabel={Время, ns}, ylabel={Длина текста, строки}]
\addplot coordinates {
    (14799, 1000)
    (198101, 10000)
    (1871960, 100000)
    (18596700, 1000000)
    (181521000, 10000000)
};
\end{axis}
\end{tikzpicture}


\newpage
\subsection*{Выводы}


Я создал функцию для поиска слова в тексте, используя Z-функцию. 
Однако, в некоторых сценариях данная функция может немного замедляться,
 особенно если в тексте много совпадений или много коротких слов.
\end{document}
